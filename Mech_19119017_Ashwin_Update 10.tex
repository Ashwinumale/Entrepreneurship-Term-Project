\documentclass{article}

\usepackage[margin=1.5in]{geometry}
\usepackage[dvipsnames]{xcolor}

\title{Finances for "Sadabahar"}
\author{Ashwin Sudhir Umale, 19119017}
\date{\today}

\begin{document}
\maketitle


\section{Introduction}

Finances of this venture would be exciting to do. Initially it would be very minimal as we need to gain popularity.
\\
In this term project I will be discussing different financial models. After that I'll be looking for a model which could be used for our venture "Sadabahar".
\\
Sadabahar is a one stop solution for multiple plant problems. We have a wide range of products and services which range from plants to their pots and from gifting to packaging. Our unique selling point is that we provide a connectivity between customers and experts, in order to help our customer whenever they need it.
\section{Parts of Finance}
Along with marketing, management, human resources, information technology, and production management, business finance is one of the functional areas of business. Finance is essentially the lifeblood of enterprises, hence it is one of the most crucial functional areas. The finance function, which is linked to accounting, is in charge of transferring funds to the business's other operating departments.
Financing has different parts like-
\begin{itemize}
    \item Budgeting
    \item Financial Analysis
    \item Investments
    \item Saving 
    \item Borrowing
\end{itemize}

\section{Financial Model}
A financial model is a description of a firm's performance based on a set of characteristics that assists the company in forecasting future financial performance. In other words, it allows a business to evaluate the financial consequences of a decision in quantitative terms. Knowledge of the company's operations, accounting, corporate finance, and Excel spreadsheets are among the measurements and abilities required to build the model.

We can go through how we'd handle extraordinary occurrences if they happen by creating a model that incorporates the business impact. Creating a model to assist we understand how we'd react to predictable changes will give us a leg up on dealing with an unexpected situation.

\section{Types of Financial Models}
Small businesses with a long history might construct data models by combining historical financial data with information from industry and market studies. Startups, on the other hand, frequently have the challenge of determining what data to utilise as the foundation of their financial models because they have little to no sales history or customer satisfaction measurements. They can receive national averages for businesses in related markets by consulting industry and market research firms like Standard and Poor's (S&P) or Dun & Bradstreet. Standard revenue costs in any industry, the proportion of revenue ascribed to direct cost of sales, and the percentage of revenue attributed to overhead are all figures that can be used.
\\
There are different types of Financial Models-
\begin{enumerate}
    \item Three Statement Model
    \item DCF Model
    \item "What-if" Analysis
    \item Industry Specific Model
    \item Strategic Forecast Model
\end{enumerate}
\section{Three Statement Model}
The three-statement model is the most important financial model and the foundation for all other financial models. The three-statement model generates a prediction for a specific time period by combining three fundamental financial statements—income statement, balance sheet, and cash flow statement—with assumptions and Excel-based calculations. It begins with revenue and goes on to compute expenses, debtors, creditors, fixed assets, and other things.

To create a clear picture of its existing business, an employee building a financial model in Excel will create tabs for the income statement (showing revenue and expenses), balance sheet (detailing assets and liabilities), cash flow statement (money in vs. money out), capital expenses, and depreciation costs. A finance expert can then utilise those historical numbers to create critical assumptions, which drive projected results, and see the forecasts using Excel-based calculations.

\section{DCF Model}
This financial model is based on the three-statement model mentioned earlier. It takes the research a step further by including a discounted cash flow (DCF) analysis to calculate the business's worth. When it comes to mergers and acquisitions, leveraged buyouts, and corporate valuations, DCF models are frequently used.
The money you expect to earn in the future must be discounted, which is where discounted cash flow analysis comes in.

The difficult element is determining what discount rate to use. Assume you have a customer who is willing to sign a five-year contract to purchase a certain amount of merchandise, and this is the foundation for your new venture. You have a sure thing if that customer has a good business—you know the company will receive a specific quantity of income for the next five years. You can invest and calculate your discounted cash flow using the interest rate on another safe asset (such as a five-year Treasury Bill).

\section{"What-if" Analysis}
This model illustrates the impact of changing assumptions such as selling price, supply chain expenses, fixed costs, expected sales, delivery costs, and other variables. In most sensitivity analysis models, one variable is changed at a time, and the impact of that modification is then demonstrated.
\\
"What-if" analysis is another name for sensitivity analysis. It forces the individual looking at the figures to think about the validity of the assumptions they've made.

\section{Industry Specific Model}
An industry-specific model might be quite complicated and difficult to understand. It simulates all of an industry's specific elements in Excel (for example, real estate, oil and gas, mining, financial institutions, ecommerce, and so on). This form of financial model necessitates a great deal of industry knowledge and experience, especially when it comes to making business input assumptions.

\section{Strategic Forecast Model}
Businesses utilise a strategic forecast model to assess how different initiatives they're considering will affect long-term, strategic goals. This technique, also known as long-range forecasting, aids firms in assessing the influence of corporate projects, treasury initiatives, and marketing and analysis plans on their long-term strategy. A corporation might use the strategic forecast model to anticipate the expenses and possible income of developing a second manufacturing plant, opening stores in a different country, or introducing a new product line, for example. It can then decide whether or not pursuing those ideas is in the best interests of the company.
\end{document}