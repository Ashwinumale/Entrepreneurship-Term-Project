\documentclass{article}

\usepackage[margin=1.5in]{geometry}
\usepackage[dvipsnames]{xcolor}

\title{Finances for "Sadabahar"}
\author{Ashwin Sudhir Umale, 19119017}
\date{\today}

\begin{document}
\maketitle


\section{Introduction}

Finances of this venture would be exciting to do. Initially it would be very minimal as we need to gain popularity.
\\
In this term project I will be discussing different financial models. After that I'll be looking for a model which could be used for our venture "Sadabahar".
\\
Sadabahar is a one stop solution for multiple plant problems. We have a wide range of products and services which range from plants to their pots and from gifting to packaging. Our unique selling point is that we provide a connectivity between customers and experts, in order to help our customer whenever they need it.
\section{Parts of Finance}
Along with marketing, management, human resources, information technology, and production management, business finance is one of the functional areas of business. Finance is essentially the lifeblood of enterprises, hence it is one of the most crucial functional areas. The finance function, which is linked to accounting, is in charge of transferring funds to the business's other operating departments.
Financing has different parts like-
\begin{itemize}
    \item Budgeting
    \item Financial Analysis
    \item Investments
    \item Saving 
    \item Borrowing
\end{itemize}

\section{Financial Model}
A financial model is a description of a firm's performance based on a set of characteristics that assists the company in forecasting future financial performance. In other words, it allows a business to evaluate the financial consequences of a decision in quantitative terms. Knowledge of the company's operations, accounting, corporate finance, and Excel spreadsheets are among the measurements and abilities required to build the model.







\end{document}